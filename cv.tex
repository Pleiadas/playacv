% cv.tex %%%%%%
% This is a CV settings and contents template for PlayaCV, based on yet another LaTeX CV (AltaCV)
% created by Martin Knöfel (https://github.com/Pleiadas/), v1.0, Nov 5, 2023
% may be distributed and/or modified under the 
% Creative Commons (https://creativecommons.org/licenses/by/4.0/) license (attribution)
% version 4.0 as of Nov 5, 2023, or under the 
% LaTeX Project Public License v1.3 (see below)
% Compiles with pdfLaTeX, XeLaTeX and LuaLaTeX
%
% To be used according to cover letter body or see ReadMe.md 
%
% PlayaCV uses the fontawesome5 icon package
% See https://texdoc.net/pkg/fontawesome5 for full list of symbols
% 
%% The files playacv.cls and cv.tex include parts from
%% altacv.cls v1.6.5, 3 Nov 2022 written by LianTze Lim (liantze@gmail.com)
%% (https://github.com/liantze/AltaCV)
%% which may be distributed and/or modified under the
%% conditions of the LaTeX Project Public License, either version 1.3
%% of this license or (at your option) any later version.
%% The latest version of this license is in http://www.latex-project.org/lppl.txt
%% and version 1.3 or later is part of all distributions of LaTeX
%% version 2003/12/01 or later.
%%%%%%%%%%%%%%%

% you might set up this file once with all your default information and settings

% your contact data
\newcommand{\myName}{Colum Namvitae}
\newcommand{\myMail}{colum.namvitae@mail.kom}
\newcommand{\myPhone}{+987 654 3210-13}
\newcommand{\myStreet}{Way 32}
\newcommand{\myZIP}{ZIP}
\newcommand{\myCity}{Aplace}
\newcommand{\myState}{Earth}
\newcommand{\myAddress}{\myStreet, \myZIP{} \myCity, \myState}
\newcommand{\myLinkedin}{my-linkedin}
% add an address link here and other urls in the cv, but don't forget to escape or encode the right characters!
\newcommand{\myAddressURL}{https://www.openstreetmap.org/\#map=5/69/42}

% change the page layout if you need to
\geometry{left=1.5cm,right=2cm,top=1.5cm,bottom=1.5cm,columnsep=0.3cm}

% change the font if you want to, depending on whether
% you're using pdflatex or xelatex/lualatex
\ifxetexorluatex
% If using xelatex or lualatex:
\setmainfont{Roboto Slab}
\setsansfont{Lato}
\renewcommand{\familydefault}{\sfdefault}
\else
% If using pdflatex:
\usepackage[rm]{roboto}
\usepackage[defaultsans]{lato}
% \usepackage{sourcesanspro}
\renewcommand{\familydefault}{\sfdefault}
\fi

% change the colours if you want to, grayscales are recommended
\definecolor{LightGrey}{HTML}{777777}
\definecolor{DarkGrey}{HTML}{333333}
\definecolor{Grey}{HTML}{424242}

% colours settings
\colorlet{name}{DarkGrey} % Name at top
\colorlet{heading}{Grey} % TITLES
\colorlet{emphasis}{black} % Titles
\colorlet{body}{DarkGrey} % Body
\colorlet{headingrule}{RuleColor} % Lines, RuleColor defined in your own name.tex cv file
\colorlet{accent}{SymbolColor} % Symbols, SymbolColor defined in your own name.tex cv file
\colorlet{highlight}{HighlightColor} % Symbols, SymbolColor defined in your own name.tex cv file

% change some fonts, if necessary
\renewcommand{\namefont}{\Huge\rmfamily\bfseries}
\renewcommand{\personalinfofont}{\footnotesize}
\renewcommand{\cvsectionfont}{\LARGE\rmfamily\bfseries}
\renewcommand{\cvsubsectionfont}{\large\bfseries}

% depending on your tastes, change fontsize of itemize environment content
\AtBeginEnvironment{itemize}{\small}

% change the bullets for itemize and rating marker
% for \cvskill if you want to
%\renewcommand{\itemmarker}{{\small\textbullet}}
%\renewcommand{\ratingmarker}{\faStar}

% add text under name (removing it might change the vertical alignment of things), 1 = include, 9 = nein    
\newcommand{\HeaderQuote}{1} % set to 1 include text under name
\newcommand{\aHeaderQuote}{PlayaCV \LaTeX{} Template} % change default text under name, e.g. Résumé for Company X, position description, your motto
\newcommand{\bHeaderQuote}{DarkGrey} % change colour of text under name

% default document settings

% header settings, 1 = include, 9 = no
% go to cvHeader.tex or to \makecvheader in playacv.cls for further modifications to the top part of the cv
\newcommand{\WithHeader}{1}

% you can add one or multiple photos on the left or right with ',' e.g. qr,pic
\newcommand{\aHeaderPhoto}{img/qr} % photo(s) file name/s without extensions, e.g. pic or qr,pic
\newcommand{\bHeaderPhoto}{2} % photo dimensions in cm

% choose one
\newcommand{\HeaderPhotoR}{1} % set to 1 to include photo(s) on the right
\newcommand{\HeaderPhotoL}{9} % set to 1 to include photo(s) on the left

% add centered watermark to header
\newcommand{\HeaderWatermark}{9} % set to 1 to include watermark
\newcommand{\aHeaderWatermark}{img/watermark} % watermark picture file name
\newcommand{\bHeaderWatermark}{2} % watermark width in cm
\newcommand{\cHeaderWatermark}{1} % watermark opacity in 0..1
\newcommand{\dHeaderWatermark}{0.0} % left -/+ right distance from center in cm
\newcommand{\eHeaderWatermark}{-0.75} % up -/+ down position in cm

% cover letter header width for left and right minipages (contact data) at the top of the page
\newcommand{\HeaderWidthL}{0.5}
\newcommand{\HeaderWidthR}{0.3}

% footer settings
\newcommand{\WithFooter}{1}

% you can add photos with text under it on the left and/or right
\newcommand{\FooterPhotoL}{9} % set to 1 to include picture on the left
\newcommand{\aFooterPhotoL}{img/qr} % picture file name without extension, e.g. qr
\newcommand{\bFooterPhotoL}{2} % picture width in cm
\newcommand{\cFooterPhotoL}{\myCity, \today} % text in the left bottom corner (under left picture), can be empty
\newcommand{\dFooterPhotoL}{.5} % distance of the bottom left picture from left border in cm
\newcommand{\eFooterPhotoL}{0.0} % distance of the left bottom text from left border in cm

\newcommand{\FooterPhotoR}{1} % set to 1 to include picture on the right, appropriate spot for signature
\newcommand{\aFooterPhotoR}{img/sign} % picture file name without extension, e.g. signature
\newcommand{\bFooterPhotoR}{4} % picture width in cm
\newcommand{\cFooterPhotoR}{\myName} % text in the right bottom corner (under right picture), can be empty
\newcommand{\dFooterPhotoR}{0.0} % distance of the bottom right picture from right border in cm
\newcommand{\eFooterPhotoR}{0.75} % distance of the right bottom text from right border in cm

% add centered watermark to footer
\newcommand{\FooterWatermark}{9} % set to 1 to include watermark
\newcommand{\aFooterWatermark}{img/watermark} % watermark picture file name
\newcommand{\bFooterWatermark}{2} % watermark dimensions in cm
\newcommand{\cFooterWatermark}{0.75} % watermark opacity in 0..1
\newcommand{\dFooterWatermark}{0.0} % left -/+ right distance from center
\newcommand{\eFooterWatermark}{0.0} % up -/+ down shift

% column settings
% set the left/right column width ratio to 37:63.
\newcommand{\MyColumnRatio}{0.4}

%%% choose default order of chapters to be displayed on the left side, from top to bottom
%%% left column

\newcommand{\LeftColumn}{
	\cvsetting{\WithSummary}{ % set to 1 to include summary chapter
	
	% uses tcolorbox for optional text background with \cvbox
	%
	%example with more options:
	%\usepackage[most]{tcolorbox}
	%
	%\tcbset{
		%    frame code={}
		%    center title,
		%    left=0pt,
		%    right=0pt,
		%    top=0pt,
		%    bottom=0pt,
		%    colback=gray!70, \
		%    colframe=white,
		%    width=\dimexpr\textwidth\relax,
		%    enlarge left by=0mm,
		%    boxsep=5pt,
		%    arc=0pt,outer arc=0pt,
		%    }
	% _Summary chapter
	\cvsection{\Summary}
	\vspace{-.25cm}
	\cvbox[\eSummary]{\aSummary}{\bSummary}{\cSummary}{\dSummary}
	\vspace{-.33cm}
	\smallskip
}
	\cvsetting{\WithPersonalInfo}{ % set to 1 to include personal information chapter
	
	% _PersonalInfo chapter
	\cvsection{\PersonalInfo}
	\cvsetting{\PersonalInfoOne}{\cvachievement{\bPersonalInfoOne}{\aPersonalInfoOne}{\cPersonalInfoOne}}
	\cvsetting{\PersonalInfoTwo}{\cvachievement{\bPersonalInfoTwo}{\aPersonalInfoTwo}{\cPersonalInfoTwo}}
	\cvsetting{\PersonalInfoThree}{\cvachievement{\bPersonalInfoThree}{\aPersonalInfoThree}{\cPersonalInfoThree}}
	\cvsetting{\PersonalInfoFour}{\cvachievement{\bPersonalInfoFour}{\aPersonalInfoFour}{\cPersonalInfoFour}}
	\cvsetting{\PersonalInfoFive}{\cvachievement{\bPersonalInfoFive}{\aPersonalInfoFive}{\cPersonalInfoFive}}
	\cvsetting{\PersonalInfoSix}{\cvachievement{\bPersonalInfoSix}{\aPersonalInfoSix}{\cPersonalInfoSix}}
	\cvsetting{\PersonalInfoSeven}{\cvachievement{\bPersonalInfoSeven}{\aPersonalInfoSeven}{\cPersonalInfoSeven}}
	\cvsetting{\PersonalInfoEight}{\cvachievement{\bPersonalInfoEight}{\aPersonalInfoEight}{\cPersonalInfoEight}}
	\cvsetting{\PersonalInfoNine}{\cvachievement{\bPersonalInfoNine}{\aPersonalInfoNine}{\cPersonalInfoNine}}
	\cvsetting{\PersonalInfoTen}{\cvachievement{\bPersonalInfoTen}{\aPersonalInfoTen}{\cPersonalInfoTen}}
}
	\input{chap/_Education2}
	\cvsetting{\WithCertifications}{ % set to 1 to include certifications chapter
	
	% _Certifications chapter
	\cvsection{\Certifications}
	\cvsetting{\CertificationOne}{
		\cventry[\fCertificationOne]{\aCertificationOne}{\bCertificationOne}{\cCertificationOne}{\dCertificationOne}
		{\cvsetting[\cvcontentfont]{\fCertificationOne}{\large}\eCertificationOne}\par
	}
	\cvsetting{\CertificationTwo}{
		\divider\\
		\cventry[\fCertificationTwo]{\aCertificationTwo}{\bCertificationTwo}{\cCertificationTwo}{\dCertificationTwo}
		{\cvsetting[\cvcontentfont]{\fCertificationTwo}{\large}\eCertificationTwo}\par
	}
	\cvsetting{\CertificationThree}{
		\divider\\
		\cventry[\fCertificationThree]{\aCertificationThree}{\bCertificationThree}{\cCertificationThree}{\dCertificationThree}
		{\cvsetting[\cvcontentfont]{\fCertificationThree}{\large}\eCertificationThree}\par
	}
	\cvsetting{\CertificationFour}{
		\divider\\
		\cventry[\fCertificationFour]{\aCertificationFour}{\bCertificationFour}{\cCertificationFour}{\dCertificationFour}
		{\cvsetting[\cvcontentfont]{\fCertificationFour}{\large}\eCertificationFour}\par
	}
	\cvsetting{\CertificationFive}{
		\divider\\
		\cventry[\fCertificationFive]{\aCertificationFive}{\bCertificationFive}{\cCertificationFive}{\dCertificationFive}
		{\cvsetting[\cvcontentfont]{\fCertificationFive}{\large}\eCertificationFive}\par
	}
	\par
	\smallskip
}
	\cvsetting{\WithSkills}{ % set to 1 to include hard skills chapter
	
	% _Skills chapter
	\cvsection{\HardSkills}
	\cvsetting{\SkillOne}{\cvskill[\fSkillOne]{\aSkillOne}{\bSkillOne}}
	\cvsetting{\SkillTwo}{\cvskill[\fSkillTwo]{\aSkillTwo}{\bSkillTwo}}
	\cvsetting{\SkillThree}{\cvskill[\fSkillThree]{\aSkillThree}{\bSkillThree}}
	\cvsetting{\SkillFour}{\cvskill[\fSkillFour]{\aSkillFour}{\bSkillFour}}
	\cvsetting{\SkillFive}{\cvskill[\fSkillFive]{\aSkillFive}{\bSkillFive}}
	\cvsetting{\SkillSix}{\cvskill[\fSkillSix]{\aSkillSix}{\bSkillSix}}
	\cvsetting{\SkillSeven}{\cvskill[\fSkillSeven]{\aSkillSeven}{\bSkillSeven}}
	\cvsetting{\SkillEight}{\cvskill[\fSkillEight]{\aSkillEight}{\bSkillEight}}
	\cvsetting{\SkillNine}{\cvskill[\fSkillNine]{\aSkillNine}{\bSkillNine}}
	\cvsetting{\SkillTen}{\cvskill[\fSkillTen]{\aSkillTen}{\bSkillTen}}
	\par
	\smallskip
}
	\cvsetting{\WithLanguages}{ % set to 1 to include languages chapter
	
	% _Languages chapter
	\cvsection{\Languages}
	\cvsetting{\LanguageOne}{\cvskill[\fLanguageOne]{\aLanguageOne \textcolor{LightGrey}{\textmd{\textsl{ \cLanguageOne}}}}{\bLanguageOne}}
	\cvsetting{\LanguageTwo}{\cvskill[\fLanguageTwo]{\aLanguageTwo \textcolor{LightGrey}{\textmd{\textsl{ \cLanguageTwo}}}}{\bLanguageTwo}}
	\cvsetting{\LanguageThree}{\cvskill[\fLanguageThree]{\aLanguageThree \textcolor{LightGrey}{\textmd{\textsl{ \cLanguageThree}}}}{\bLanguageThree}}
	\cvsetting{\LanguageFour}{\cvskill[\fLanguageFour]{\aLanguageFour \textcolor{LightGrey}{\textmd{\textsl{ \cLanguageFour}}}}{\bLanguageFour}}
	\cvsetting{\LanguageFive}{\cvskill[\fLanguageFive]{\aLanguageFive \textcolor{LightGrey}{\textmd{\textsl{ \cLanguageFive}}}}{\bLanguageFive}}
	\cvsetting{\LanguageSix}{\cvskill[\fLanguageSix]{\aLanguageSix \textcolor{LightGrey}{\textmd{\textsl{ \cLanguageSix}}}}{\bLanguageSix}}
	\cvsetting{\LanguageSeven}{\cvskill[\fLanguageSeven]{\aLanguageSeven \textcolor{LightGrey}{\textmd{\textsl{ \cLanguageSeven}}}}{\bLanguageSeven}}
	\cvsetting{\LanguageEight}{\cvskill[\fLanguageEight]{\aLanguageEight \textcolor{LightGrey}{\textmd{\textsl{ \cLanguageEight}}}}{\bLanguageEight}}
	\cvsetting{\LanguageNine}{\cvskill[\fLanguageNine]{\aLanguageNine \textcolor{LightGrey}{\textmd{\textsl{ \cLanguageNine}}}}{\bLanguageNine}}
	\cvsetting{\LanguageTen}{\cvskill[\fLanguageTen]{\aLanguageTen \textcolor{LightGrey}{\textmd{\textsl{ \cLanguageTen}}}}{\bLanguageTen}}
	\par
	\smallskip
}
	
	% you can force a page break within a column with \newpage
	%\newpage
	
	\cvsetting{\WithAwards}{ % set to 1 to include awards/honors chapter
	
	% _Awards chapter
	\cvsection{\Awards}
	\cvsetting{\AwardOne}{
		\cvachievement[\fAwardOne]{\bAwardOne}{\aAwardOne}{{\cvsetting[\cvcontentfont]{\fAwardOne}{\cvhighlightfont} \cAwardOne}}
	}
	\cvsetting{\AwardTwo}{
		\divider\\
		\cvachievement[\fAwardTwo]{\bAwardTwo}{\aAwardTwo}{{\cvsetting[\cvcontentfont]{\fAwardTwo}{\cvhighlightfont} \cAwardTwo}}
	}
	\cvsetting{\AwardThree}{
		\divider\\
		\cvachievement[\fAwardThree]{\bAwardThree}{\aAwardThree}{{\cvsetting[\cvcontentfont]{\fAwardThree}{\cvhighlightfont} \cAwardThree}}
	}
	\cvsetting{\AwardFour}{
		\divider\\
		\cvachievement[\fAwardFour]{\bAwardFour}{\aAwardFour}{{\cvsetting[\cvcontentfont]{\fAwardFour}{\cvhighlightfont} \cAwardFour}}
	}
	\cvsetting{\AwardFive}{
		\divider\\
		\cvachievement[\fAwardFive]{\bAwardFive}{\aAwardFive}{{\cvsetting[\cvcontentfont]{\fAwardFive}{\cvhighlightfont} \cAwardFive}}
	}
	\par
	\smallskip
}
	\cvsetting{\WithSoftSkills}{ % set to 1 to include soft skills chapter
	
	% _SoftSkills chapter
	\cvsection{\SoftSkills}
	% Use skill levels if you want to, you need \bSoftSkill.. set then
%		\cvsetting{\SoftSkillOne}{\cvSoftSkill[\fSoftSoftSkillOne]{\aSoftSoftSkillOne}{\bSoftSoftSkillOne}}
%		\cvsetting{\SoftSkillTwo}{\cvSoftSkill[\fSoftSkillTwo]{\aSoftSkillTwo}{\bSoftSkillTwo}}
%		\cvsetting{\SoftSkillThree}{\cvSoftSkill[\fSoftSkillThree]{\aSoftSkillThree}{\bSoftSkillThree}}
%		\cvsetting{\SoftSkillFour}{\cvSoftSkill[\fSoftSkillFour]{\aSoftSkillFour}{\bSoftSkillFour}}
%		\cvsetting{\SoftSkillFive}{\cvSoftSkill[\fSoftSkillFive]{\aSoftSkillFive}{\bSoftSkillFive}}
%		\cvsetting{\SoftSkillSix}{\cvSoftSkill[\fSoftSkillSix]{\aSoftSkillSix}{\bSoftSkillSix}}
%		\cvsetting{\SoftSkillSeven}{\cvSoftSkill[\fSoftSkillSeven]{\aSoftSkillSeven}{\bSoftSkillSeven}}
%		\cvsetting{\SoftSkillEight}{\cvSoftSkill[\fSoftSkillEight]{\aSoftSkillEight}{\bSoftSkillEight}}
%		\cvsetting{\SoftSkillNine}{\cvSoftSkill[\fSoftSkillNine]{\aSoftSkillNine}{\bSoftSkillNine}}
%		\cvsetting{\SoftSkillTen}{\cvSoftSkill[\fSoftSkillTen]{\aSoftSkillTen}{\bSoftSkillTen}}
	\begin{minipage}{\columnwidth}
		\cvsetting{\SoftSkillOne}{\cvtag[\fSoftSkillOne]{\aSoftSkillOne}}\hfill\cvsetting{\SoftSkillTwo}{\cvtag[\fSoftSkillTwo]{\aSoftSkillTwo}}\hfill\cvsetting{\SoftSkillThree}{\cvtag[\fSoftSkillThree]{\aSoftSkillThree}}\hfill\cvsetting{\SoftSkillFour}{\cvtag[\fSoftSkillFour]{\aSoftSkillFour}}\hfill\cvsetting{\SoftSkillFive}{\cvtag[\fSoftSkillFive]{\aSoftSkillFive}}\hfill\cvsetting{\SoftSkillSix}{\cvtag[\fSoftSkillSix]{\aSoftSkillSix}}\hfill\cvsetting{\SoftSkillSeven}{\cvtag[\fSoftSkillSeven]{\aSoftSkillSeven}}\hfill\cvsetting{\SoftSkillEight}{\cvtag[\fSoftSkillEight]{\aSoftSkillEight}}\hfill\cvsetting{\SoftSkillNine}{\cvtag[\fSoftSkillNine]{\aSoftSkillNine}}\hfill\cvsetting{\SoftSkillTen}{\cvtag[\fSoftSkillTen]{\aSoftSkillTen}}\hfill\cvsetting{\SoftSkillEleven}{\cvtag[\fSoftSkillEleven]{\aSoftSkillEleven}}\hfill\cvsetting{\SoftSkillTwelve}{\cvtag[\fSoftSkillTwelve]{\aSoftSkillTwelve}}\hfill\cvsetting{\SoftSkillThirteen}{\cvtag[\fSoftSkillThirteen]{\aSoftSkillThirteen}}\hfill\cvsetting{\SoftSkillFourteen}{\cvtag[\fSoftSkillFourteen]{\aSoftSkillFourteen}}\hfill\cvsetting{\SoftSkillFifteen}{\cvtag[\fSoftSkillFifteen]{\aSoftSkillFifteen}}
	\end{minipage}
	\par
	\smallskip
}

	\cvsetting{\WithProjects}{ % set to 1 to include projects chapter
	
	% _Projects chapter
	\cvsection{\Projects}
	\cvevent{\aProjectOne}{\bProjectOne}{\dProjectOne}{\cProjectOne}
	\eProjectOne\par
	\cvsetting{\ProjectTwo}{
		\vspace{-.25cm}
		\divider
		\cvevent{\aProjectTwo}{\bProjectTwo}{\dProjectTwo}{\cProjectTwo}
		\eProjectTwo\par
	}
	\cvsetting{\ProjectThree}{
		\vspace{-.25cm}
		\divider
		\cvevent{\aProjectThree}{\bProjectThree}{\dProjectThree}{\cProjectThree}
		\eProjectThree\par
	}
	\cvsetting{\ProjectFour}{
		\vspace{-.25cm}
		\divider
		\cvevent{\aProjectFour}{\bProjectFour}{\dProjectFour}{\cProjectFour}
		\eProjectFour\par
	}
	\cvsetting{\ProjectFive}{
		\vspace{-.25cm}
		\divider
		\cvevent{\aProjectFive}{\bProjectFive}{\dProjectFive}{\cProjectFive}
		\eProjectFive\par
	}
	
	\smallskip
}
	\cvsetting{\WithReferences}{ % set to 1 to include references chapter
	
	% _References chapter
	\cvsection{\References}
	\cvsetting{\ReferenceOne}{
		\aReferenceOne}
	\cvsetting{\ReferenceTwo}{
		\divider\\
		\smallskip
		\aReferenceTwo}
	\cvsetting{\ReferenceThree}{
		\divider\\
		\smallskip
		\aReferenceThree}
	\cvsetting{\ReferenceFour}{
		\divider\\
		\smallskip
		\aReferenceFour}
	\cvsetting{\ReferenceFive}{\aReferenceFive}
	\divider\\
	\href{mailto:\myMail}{\faEnvelope[]}\ReferenceText
	\par
	\smallskip
}

	
	% rogue chapter
	%\cvsection{Rogue life} % feel free to add a new blank chapter in any column with \cvsection{MyTitle}
	%feel free to add a new blank chapter with \texttt{cvsection$\textmd{\{Rogue chapter\}}$} in any column
	%\smallskip
}

%%% choose default order of chapters to be displayed on the right side, from top to bottom
%%% right column
%
\newcommand{\RightColumn}{
	\cvsetting{\WithExperience}{ % set to 1 to include experience chapter
	
	% _Experience chapter
	\cvsection{\Experience}
	\cvsetting{\ExperienceOne}{
		\adExperienceOne
		\cvsetting[\cvsetting{\fExperienceOne}{\large} \eExperienceOne]{\eExperienceItemOne}{\eExperienceOne
			\eExperienceItemsOne}\par
	}
	\cvsetting{\ExperienceTwo}{
		\vspace{-.25cm}
		\divider\\
		\adExperienceTwo
		\cvsetting[\cvsetting{\fExperienceTwo}{\large} \eExperienceTwo]{\eExperienceItemTwo}{\eExperienceTwo
			\eExperienceItemsTwo}\par
	}
	\cvsetting{\ExperienceThree}{
		\vspace{-.25cm}
		\divider\\
		\adExperienceThree
		\cvsetting[\cvsetting{\fExperienceThree}{\large} \eExperienceThree]{\eExperienceItemThree}{\eExperienceThree
			\eExperienceItemsThree}\par
	}
	\cvsetting{\ExperienceFour}{
		\vspace{-.25cm}
		\divider\\
		\adExperienceFour
		\cvsetting[\cvsetting{\fExperienceFour}{\large} \eExperienceFour]{\eExperienceItemFour}{\eExperienceFour
			\eExperienceItemsFour}\par
	}
	\cvsetting{\ExperienceFive}{
		\vspace{-.25cm}
		\divider\\
		\adExperienceFive
		\cvsetting[\cvsetting{\fExperienceFive}{\large} \eExperienceFive]{\eExperienceItemFive}{\eExperienceFive
			\eExperienceItemsFive}\par
	}
	\cvsetting{\ExperienceSix}{
		\vspace{-.25cm}
		\divider\\
		\adExperienceSix
		\cvsetting[\cvsetting{\fExperienceSix}{\large} \eExperienceSix]{\eExperienceItemSix}{\eExperienceSix
			\eExperienceItemsSix}\par
	}
	\cvsetting{\ExperienceSeven}{
		\vspace{-.25cm}
		\divider\\
		\adExperienceSeven
		\cvsetting[\cvsetting{\fExperienceSeven}{\large} \eExperienceSeven]{\eExperienceItemSeven}{\eExperienceSeven
			\eExperienceItemsSeven}\par
	}
	\cvsetting{\ExperienceEight}{
		\vspace{-.25cm}
		\divider\\
		\adExperienceEight
		\cvsetting[\cvsetting{\fExperienceEight}{\large} \eExperienceEight]{\eExperienceItemEight}{\eExperienceEight
			\eExperienceItemsEight}\par
	}
	\cvsetting{\ExperienceNine}{
		\vspace{-.25cm}
		\divider\\
		\adExperienceNine
		\cvsetting[\cvsetting{\fExperienceNine}{\large} \eExperienceNine]{\eExperienceItemNine}{\eExperienceNine
			\eExperienceItemsNine}\par
	}
	\cvsetting{\ExperienceTen}{
		\vspace{-.25cm}
		\divider\\
		\adExperienceTen
		\cvsetting[\cvsetting{\fExperienceTen}{\large} \eExperienceTen]{\eExperienceItemTen}{\eExperienceTen
			\eExperienceItemsTen}\par
	}
	\smallskip
}

	\cvsetting{\WithEducation}{ % set to 1 to include education chapter
	
	% _Education chapter
	\cvsection{\Education}
	\cvsetting{\EducationOne}{
		\cventry{\aEducationOne}{\bEducationOne}{\cEducationOne}{\dEducationOne}
		\cvsetting[\cvsetting{\fEducationOne}{\large} \eEducationOne]{\eEducationItemOne}{\eEducationOne
			\eEducationItemsOne}\par
	}
	\cvsetting{\EducationTwo}{
		\vspace{-.25cm}
		\divider\\
		\cventry{\aEducationTwo}{\bEducationTwo}{\cEducationTwo}{\dEducationTwo}
		\cvsetting[\cvsetting{\fEducationTwo}{\large} \eEducationTwo]{\eEducationItemTwo}{\eEducationTwo
			\eEducationItemsTwo}\par
	}
	\cvsetting{\EducationThree}{
		\vspace{-.25cm}
		\divider\\
		\cventry{\aEducationThree}{\bEducationThree}{\cEducationThree}{\dEducationThree}
		\cvsetting[\cvsetting{\fEducationThree}{\large} \eEducationThree]{\eEducationItemThree}{\eEducationThree
			\eEducationItemsThree}\par
	}
	\cvsetting{\EducationFour}{
		\vspace{-.25cm}
		\divider\\
		\cventry{\aEducationFour}{\bEducationFour}{\cEducationFour}{\dEducationFour}
		\cvsetting[\cvsetting{\fEducationFour}{\large} \eEducationFour]{\eEducationItemFour}{\eEducationFour
			\eEducationItemsFour}\par
	}
	\cvsetting{\EducationFive}{
		\vspace{-.25cm}
		\divider\\
		\cventry{\aEducationFive}{\bEducationFive}{\cEducationFive}{\dEducationFive}
		\cvsetting[\cvsetting{\fEducationFive}{\large} \eEducationFive]{\eEducationItemFive}{\eEducationFive
			\eEducationItemsFive}\par
	}
	\smallskip
}
	\cvsetting{\WithPublications}{ % set to 1 to include publications chapter
	
	% _Publications chapter
	\cvsection{\Publications}
	\cvsetting{\PublicationOne}{ %
		\cventry[\fPublicationOne]{\aPublicationOne}{\bPublicationOne}{\faGlobe{} \cPublicationOne}{\faCalendar{} \dPublicationOne}
		\cvsetting[\cvsetting{\fPublicationOne}{\large} \ePublicationOne]{\ePublicationItemOne}{\ePublicationOne
			\ePublicationItemsOne}\par
	}
	\cvsetting{\PublicationTwo}{ %
		\vspace{-.25cm}
		\smallskip
		\divider\\
		\cventry[\fPublicationTwo]{\aPublicationTwo}{\bPublicationTwo}{\faGlobe{} \cPublicationTwo}{\faCalendar{} \dPublicationTwo}
		\cvsetting[\cvsetting{\fPublicationTwo}{\large} \ePublicationTwo]{\ePublicationItemTwo}{\ePublicationTwo
			\ePublicationItemsTwo}\par
	}
	\cvsetting{\PublicationThree}{ %
		\vspace{-.25cm}
		\divider\\
		\cventry[\fPublicationThree]{\aPublicationThree}{\bPublicationThree}{\faGlobe{} \cPublicationThree}{\faCalendar{} \dPublicationThree}
		\cvsetting[\cvsetting{\fPublicationThree}{\large} \ePublicationThree]{\ePublicationItemThree}{\ePublicationThree
			\ePublicationItemsThree}\par
	}
	\cvsetting{\PublicationFour}{ %
		\vspace{-.25cm}
		\divider\\
		\cventry[\fPublicationFour]{\aPublicationFour}{\bPublicationFour}{\faGlobe{} \cPublicationFour}{\faCalendar{} \dPublicationFour}
		\cvsetting[\cvsetting{\fPublicationFour}{\large} \ePublicationFour]{\ePublicationItemFour}{\ePublicationFour
			\ePublicationItemsFour}\par
	}
	\cvsetting{\PublicationFive}{ %
		\vspace{-.25cm}
		\divider\\
		\cventry[\fPublicationFive]{\aPublicationFive}{\bPublicationFive}{\faGlobe{} \cPublicationFive}{\faCalendar{} \dPublicationFive}
		\cvsetting[\cvsetting{\fPublicationFive}{\large} \ePublicationFive]{\ePublicationItemFive}{\ePublicationFive
			\ePublicationItemsFive}\par
	}
	\smallskip
}
	\cvsetting{\WithSocial}{ % set to 1 to include volunteering chapter
	
	% _Volunteering chapter
	\cvsection{\Social}
		\cvsetting{\SocialOne}{
		\adSocialOne
		\cvsetting[\cvsetting{\fSocialOne}{\large} \eSocialOne]{\eSocialItemOne}{\eSocialOne
			\eSocialItemsOne}\par
	}
	\cvsetting{\SocialTwo}{
		\vspace{-.25cm}
		\divider\\
		\adSocialTwo
		\cvsetting[\cvsetting{\fSocialTwo}{\large} \eSocialTwo]{\eSocialItemTwo}{\eSocialTwo
			\eSocialItemsTwo}\par
	}
	\cvsetting{\SocialThree}{
		\vspace{-.25cm}
		\divider\\
		\adSocialThree
		\cvsetting[\cvsetting{\fSocialThree}{\large} \eSocialThree]{\eSocialItemThree}{\eSocialThree
			\eSocialItemsThree}\par
	}
	\cvsetting{\SocialFour}{
		\vspace{-.25cm}
		\divider\\
		\adSocialFour
		\cvsetting[\cvsetting{\fSocialFour}{\large} \eSocialFour]{\eSocialItemFour}{\eSocialFour
			\eSocialItemsFour}\par
	}
	\cvsetting{\SocialFive}{
		\vspace{-.25cm}
		\divider\\
		\adSocialFive
		\cvsetting[\cvsetting{\fSocialFive}{\large} \eSocialFive]{\eSocialItemFive}{\eSocialFive
			\eSocialItemsFive}\par
	}
	\smallskip
}
	\cvsetting{\WithHobbies}{ % set to 1 to include hobbies chapter
	
	% _Hobbies chapter
	\cvsection{\Hobbies}
	\begin{minipage}{\columnwidth}
		\centering
\cvsetting{\HobbyOne}{\cvtag{\aHobbyOne}}\hfill\cvsetting{\HobbyTwo}{\cvtag{\aHobbyTwo}}\hfill\cvsetting{\HobbyThree}{\cvtag{\aHobbyThree}}\hfill\cvsetting{\HobbyFour}{\cvtag{\aHobbyFour}}\hfill\cvsetting{\HobbyFive}{\cvtag{\aHobbyFive}}\hfill\cvsetting{\HobbySix}{\cvtag{\aHobbySix}}\hfill\cvsetting{\HobbySeven}{\cvtag{\aHobbySeven}}\hfill\cvsetting{\HobbyEight}{\cvtag{\aHobbyEight}}\hfill\cvsetting{\HobbyNine}{\cvtag{\aHobbyNine}}\hfill\cvsetting{\HobbyTen}{\cvtag{\aHobbyTen}}\hfill\cvsetting{\HobbyEleven}{\cvtag{\aHobbyEleven}}\hfill\cvsetting{\HobbyTwelve}{\cvtag{\aHobbyTwelve}}\hfill\cvsetting{\HobbyThirteen}{\cvtag{\aHobbyThirteen}}\hfill\cvsetting{\HobbyFourteen}{\cvtag{\aHobbyFourteen}}\hfill\cvsetting{\HobbyFifteen}{\cvtag{\aHobbyFifteen}}\par
	\end{minipage}
	\par
	\smallskip
}
	%\cvsetting{\WithUser}{ % set to 1 to include user defined chapter
	
	\cvsection{\User}
	% mychapter user defined chapter
	\cvsetting{\UserOne}{\aUserOne}
	
	% you can add your own cv elements here
	%\cvsetting[]{\UserTwo}{}
	%\cvitems[]{}{}{}{}{}
	%\cventry[]{}{}{}{}
	%\cvevent[]{}{}{}{}
	%\cvachievement[]{}{}{}
	%\cvskill[]{}{}
	%\cvtag[]{}
	%\cvref[]{}{}{}{}{}{}{}{}
	
	\smallskip
}
}

% pubs.bib contains your publications
% \addbibresource{bib/pubs.bib}

%% Use (and optionally edit if necessary) this .tex if you
%% want to use an author-year reference style like APA(6)
%% for your publication list
% % When using APA6 if you need more author names to be listed
% because you're e.g. the 12th author, add apamaxprtauth=12
\usepackage[backend=biber,style=apa6,sorting=ydnt]{biblatex}
\defbibheading{pubtype}{\cvsubsection{#1}}
\renewcommand{\bibsetup}{\vspace*{-\baselineskip}}
\AtEveryBibitem{%
	\makebox[\bibhang][l]{\itemmarker}%
	\iffieldundef{doi}{}{\clearfield{url}}%
}
\setlength{\bibitemsep}{0.25\baselineskip}
\setlength{\bibhang}{1.25em}


%% Use (and optionally edit if necessary) this .tex if you
%% want an originally numerical reference style like IEEE
%% for your publication list
% \usepackage[backend=biber,style=ieee,sorting=ydnt]{biblatex}
%% For removing numbering entirely when using a numeric style
\setlength{\bibhang}{1.25em}
\DeclareFieldFormat{labelnumberwidth}{\makebox[\bibhang][l]{\itemmarker}}
\setlength{\biblabelsep}{0pt}
\defbibheading{pubtype}{\cvsubsection{#1}}
\renewcommand{\bibsetup}{\vspace*{-\baselineskip}}
\AtEveryBibitem{%
	\iffieldundef{doi}{}{\clearfield{url}}%
}


% your default chapters and cv content settings %%%%%%%%%%%%%%%%%%%%%%%%%%% % CONTENT SETTINGS %
%
% set visibility and values. Uses 
% \newcommand{\WithCvContent}{1} here, to include items for anywhere or set default values,
% e.g. \cvsetting{\WithCvContent}{\CvThings} would include \CvThings somewhere
%
%%%%%%%%%%%%%%%%%%%%% these {setting}{values} will be used as standard values when compiling %%%

% letter default settings
\newcommand{\OnlyCoverLetter}{9} % set to 1 to print only letter to PDF
\newcommand{\WithCoverLetter}{9} % set to 1 to print letter + CV to PDF
\newcommand{\WithBody}{1}
\newcommand{\Body}{Cover letter body}

% abstract settings
\newcommand{\WithAbstract}{9}
\newcommand{\Abstract}{Abstract}

\newcommand{\aAbstract}{A short abstract in the beginning, if you wish}
\newcommand{\bAbstract}{\large}
\newcommand{\cAbstract}{accent}

% appendix settings e.g. for adding additional documents to your application
\newcommand{\WithAppendix}{9}

\newcommand{\AppendixOne}{1} % set to 1 to include Appendix chapter #1
\newcommand{\aAppendixOne}{Appendix 1 (one page pdf/image)} % appendix TITLE
\newcommand{\bAppendixOne}{appendix/snow} % appendix file (image, pdf)
\newcommand{\cAppendixOne}{0.96} % scaling factor
\newcommand{\dAppendixOne}{1} % if greater then 1, prints all pages from 1 to \dAppendixOne


\newcommand{\AppendixTwo}{1} % appendix #2
\newcommand{\aAppendixTwo}{Appendix 2 (more pages pdf file)}
\newcommand{\bAppendixTwo}{appendix/sample}
\newcommand{\cAppendixTwo}{0.95}
\newcommand{\dAppendixTwo}{3} % print the first 3 pages


\newcommand{\AppendixThree}{9} % appendix #3
\newcommand{\aAppendixThree}{Appendix 3}
\newcommand{\bAppendixThree}{appendix/..}
\newcommand{\cAppendixThree}{0.8}
\newcommand{\dAppendixThree}{3}


\newcommand{\AppendixFour}{9} % appendix #4
\newcommand{\aAppendixFour}{Appendix 4}
\newcommand{\bAppendixFour}{appendix/..}
\newcommand{\cAppendixFour}{0.8}
\newcommand{\dAppendixFour}{1}


\newcommand{\AppendixFive}{9} % appendix #5
\newcommand{\aAppendixFive}{Appendix 5}
\newcommand{\bAppendixFive}{appendix/..}
\newcommand{\cAppendixFive}{.5}
\newcommand{\dAppendixFive}{1}


% awards chapter settings
\newcommand{\WithAwards}{9} % set to 1 to include awards chapter
\newcommand{\Awards}{Awards and honors} % chapter TITLE

\newcommand{\AwardOne}{1} % set to 1 to include award/honor
\newcommand{\aAwardOne}{My awesome award} % name of award/honor
\newcommand{\bAwardOne}{\faTrophy} % symbol left of award (see https://texdoc.org/serve/fontawesome5/0)
\newcommand{\cAwardOne}{20YY The deed that brought it} % description
\newcommand{\fAwardOne}{9} % set to 1 to highlight award/honor

\newcommand{\AwardTwo}{1} % award #2
\newcommand{\aAwardTwo}{I'm a star for this}
\newcommand{\bAwardTwo}{\faStar}
\newcommand{\cAwardTwo}{20YY Something wild}
\newcommand{\fAwardTwo}{9}

\newcommand{\AwardThree}{9} % award #3
\newcommand{\aAwardThree}{Award 3}
\newcommand{\bAwardThree}{\faTrophy}
\newcommand{\cAwardThree}{YYYY}
\newcommand{\fAwardThree}{9}

\newcommand{\AwardFour}{9} % award #4
\newcommand{\aAwardFour}{Award 4}
\newcommand{\bAwardFour}{\faTrophy}
\newcommand{\cAwardFour}{YYYY}
\newcommand{\fAwardFour}{9}

\newcommand{\AwardFive}{9} % award #5
\newcommand{\aAwardFive}{Award 5}
\newcommand{\bAwardFive}{\faTrophy}
\newcommand{\cAwardFive}{YYYY}
\newcommand{\fAwardFive}{9}

% certifications chapter settings
\newcommand{\WithCertifications}{9} % set to 1 to include certifications chapter
\newcommand{\Certifications}{Certifications} % chapter TITLE

\newcommand{\CertificationOne}{1} % set to 1 to include certification
\newcommand{\aCertificationOne}{Certification 1} % create your own cv item, copy from other chapters
\newcommand{\bCertificationOne}{Issuer Name} %
\newcommand{\cCertificationOne}{Cert Number} %
\newcommand{\dCertificationOne}{12345678} %
\newcommand{\eCertificationOne}{A text} %
\newcommand{\fCertificationOne}{1} % set to 1 to highlight certification

\newcommand{\CertificationTwo}{1} % certification 2
\newcommand{\aCertificationTwo}{Certification 2}
\newcommand{\bCertificationTwo}{Issuer Name}
\newcommand{\cCertificationTwo}{Verification Number}
\newcommand{\dCertificationTwo}{12345678}
\newcommand{\eCertificationTwo}{}
\newcommand{\fCertificationTwo}{9}

\newcommand{\CertificationThree}{9} % certification 3
\newcommand{\aCertificationThree}{Certification 3}
\newcommand{\bCertificationThree}{Issuer Name}
\newcommand{\cCertificationThree}{Verification Number}
\newcommand{\dCertificationThree}{12345678}
\newcommand{\eCertificationThree}{A text}
\newcommand{\fCertificationThree}{9}

\newcommand{\CertificationFour}{9} % certification 4
\newcommand{\aCertificationFour}{Certification 4}
\newcommand{\bCertificationFour}{Issuer Name}
\newcommand{\cCertificationFour}{Verification Number}
\newcommand{\dCertificationFour}{12345678}
\newcommand{\eCertificationFour}{A text}
\newcommand{\fCertificationFour}{9}

\newcommand{\CertificationFive}{9} % certification 5
\newcommand{\aCertificationFive}{Certification 5}
\newcommand{\bCertificationFive}{Issuer Name}
\newcommand{\cCertificationFive}{Verification Number}
\newcommand{\dCertificationFive}{12345678}
\newcommand{\eCertificationFive}{A text}
\newcommand{\fCertificationFive}{9}

% education chapter settings
\newcommand{\WithEducation}{1} % set to 1 to include education chapter
\newcommand{\Education}{Education} % chapter TITLE

\newcommand{\EducationOne}{1} % set to 1 to include education entry
\newcommand{\aEducationOne}{My last degree prgramme} % plain title on the left
\newcommand{\bEducationOne}{
	\textcolor{HighlightColor}{\href{https://link/}{B.Sc./M.A./PhD in my specialty}}
} % smaller, bold, right text
\newcommand{\cEducationOne}{\faUniversity Awesome University, City, Earth} % left symbol and text
\newcommand{\dEducationOne}{\faCalendar{} Oct 20YY -- Sep 20YY} % right symbol and text e.g. date
\newcommand{\eEducationItemOne}{9} % set to 1 to show Items
\newcommand{\eEducationOne}{Thesis: Something interesting \\Grade Point Average: 3.3} % text (above Items)
\newcommand{\fEducationOne}{9} % set to 1 to highlight education entry (item)
\newcommand{\eEducationItemsOne}{
	\cvitems
	[\fEducationOne] % highlight 1..5 or 9
	{Major 1} % items {1}..{5}
	{Major 2}
	{}
	{}
	{}
}

\newcommand{\EducationTwo}{1} % education #2
\newcommand{\aEducationTwo}{Previous school} % compensate with \vspace when leaving a title out
\newcommand{\bEducationTwo}{That specialisation} 
\newcommand{\cEducationTwo}{\faMapMarker{} City, Earth}
\newcommand{\dEducationTwo}{\faCalendar{} Sep 20YY - Jun 20YY}
\newcommand{\eEducationItemTwo}{9} % set to 1 to use \eEducationItemsTwo
\newcommand{\eEducationTwo}{
	Participated in extracurricular activities
}
\newcommand{\fEducationTwo}{9}
\newcommand{\eEducationItemsTwo}{\cvitems[\fEducationTwo]{}{}{}{}{}}

\newcommand{\EducationThree}{9} % education #3
\newcommand{\aEducationThree}{Education 3}
\newcommand{\bEducationThree}{Title/Degree}
\newcommand{\cEducationThree}{\faMapMarker{} }
\newcommand{\dEducationThree}{\faCalendar{} always -- \today}
\newcommand{\eEducationItemThree}{1}
\newcommand{\eEducationThree}{A description}
\newcommand{\fEducationThree}{9}
\newcommand{\eEducationItemsThree}{\cvitems[\fEducationThree]{}{}{}{}{}}

\newcommand{\EducationFour}{9} % education #4
\newcommand{\aEducationFour}{Education 4}
\newcommand{\bEducationFour}{Title/Degree}
\newcommand{\cEducationFour}{everywhere}
\newcommand{\dEducationFour}{\faCalendar{} }
\newcommand{\eEducationItemFour}{9}
\newcommand{\eEducationFour}{A description}
\newcommand{\fEducationFour}{9}
\newcommand{\eEducationItemsFour}{\cvitems[\fEducationFour]{And an item}{}{}{}{}}

\newcommand{\EducationFive}{9} % education #5
\newcommand{\aEducationFive}{Education 5}
\newcommand{\bEducationFive}{Title/Degree}
\newcommand{\cEducationFive}{place, Earth}
\newcommand{\dEducationFive}{\faCalendar{} }
\newcommand{\eEducationItemFive}{9}
\newcommand{\eEducationFive}{}
\newcommand{\fEducationFive}{9}
\newcommand{\eEducationItemsFive}{\cvitems[\fEducationFive]{Good deed 1}{}{}{}{}}

% further education chapter settings
\newcommand{\WithTrainings}{9} % set to 1 to include further education chapter
\newcommand{\Trainings}{Further Education} % chapter TITLE

\newcommand{\TrainingsIndent}{1} % ToDo, see [!indent] in playacv.cls] 
\newcommand{\TrainingsItems}{\cvlist[\TrainingsIndent]
	{20YY}{This relevant training (6 months) $\cdot$ That bootcamp (1 week) $\cdot$ Also done (1 day)} % description item[{20YY}] {text} #1
	{Topic}{Specialisation as ..} % item #2
	{}{} % item #3
	{}{} % item #4 (add new \cvlist if you need more than 4)
}

% experience chapter settings
\newcommand{\WithExperience}{1} % set to 1 to include experience chapter
\newcommand{\Experience}{Experience} % chapter TITLE

\newcommand{\ExperienceOne}{1} % set to 1 to include experience entry
\newcommand{\adExperienceOne}{\cvevent
	{The job I wanna leave \hfill {\small \textbf{current Company}}} % optional title
	{} % optional smaller, bold, subtitle
	{Location} % location
	{May 2023 -- now} % time period
}
\newcommand{\eExperienceItemOne}{1} % set to 1 to show Items
\newcommand{\eExperienceOne}{Uses only first line} % text (above items)
\newcommand{\fExperienceOne}{3} % set to 1..5 to highlight experience or 9
\newcommand{\eExperienceItemsOne}{\cvitems[\fExperienceOne] % highlight option 1..5 or 9
	{My main responsibilities} % items {1}..{5}
	{Impressive (highlighted with fExperienceOne = 3)}
	{Item 3}
	{}
	{}
}

\newcommand{\ExperienceTwo}{1} % experience #2
\newcommand{\adExperienceTwo}{
	\cvevent{My fantastic experience} % optional title
	{Previous company} % optional subtitle
	{Beautiful city, Earth}
	{May 2021 -- Apr 2023}
}
\newcommand{\eExperienceItemTwo}{1}
\newcommand{\eExperienceTwo}{Title first line, subtitle and highlight} % own text/items
\newcommand{\fExperienceTwo}{2} % set to 1..5 to highlight experience
\newcommand{\eExperienceItemsTwo}{\cvitems[\fExperienceTwo]{Masterfully done this}{Also took care of that}{}{}{}}

\newcommand{\ExperienceThree}{1} % experience #3
\newcommand{\adExperienceThree}{
	\cvevent{} % optional title
	{My pandemic experience} % small bold title
	{\myAddress} % location
	{Mar 2020 -- Apr 2021} % time period
}
\newcommand{\eExperienceItemThree}{9}
\newcommand{\eExperienceThree}{Only bold small (sub)title}
\newcommand{\fExperienceThree}{9}
\newcommand{\eExperienceItemsThree}{\cvitems[\fExperienceThree]{}{}{}{}{}}

\newcommand{\ExperienceFour}{9} % experience #4
\newcommand{\adExperienceFour}{
	\cvevent{}{}{}{}
}
\newcommand{\eExperienceItemFour}{9}
\newcommand{\eExperienceFour}{}
\newcommand{\fExperienceFour}{9}
\newcommand{\eExperienceItemsFour}{\cvitems[\fExperienceFour]{}{}{}{}{}}

\newcommand{\ExperienceFive}{9} % experience #5
\newcommand{\adExperienceFive}{
	\cvevent{}{}{}{}
}
\newcommand{\eExperienceItemFive}{9}
\newcommand{\eExperienceFive}{}
\newcommand{\fExperienceFive}{9}
\newcommand{\eExperienceItemsFive}{\cvitems[\fExperienceFive]{}{}{}{}{}}

\newcommand{\ExperienceSix}{9} % experience #6
\newcommand{\adExperienceSix}{
	\cvevent{}{}{}{}
}
\newcommand{\eExperienceItemSix}{9}
\newcommand{\eExperienceSix}{}
\newcommand{\fExperienceSix}{9}
\newcommand{\eExperienceItemsSix}{\cvitems[\fExperienceSix]{}{}{}{}{}}

\newcommand{\ExperienceSeven}{9} % experience #7
\newcommand{\adExperienceSeven}{
	\cvevent{}{}{}{}
}
\newcommand{\eExperienceItemSeven}{9}
\newcommand{\eExperienceSeven}{}
\newcommand{\fExperienceSeven}{9}
\newcommand{\eExperienceItemsSeven}{\cvitems[\fExperienceSeven]{}{}{}{}{}}

\newcommand{\ExperienceEight}{9} % experience #8
\newcommand{\adExperienceEight}{
	\cvevent{}{}{}{}
}
\newcommand{\eExperienceItemEight}{9}
\newcommand{\eExperienceEight}{}
\newcommand{\fExperienceEight}{9}
\newcommand{\eExperienceItemsEight}{\cvitems[\fExperienceEight]{}{}{}{}{}}

\newcommand{\ExperienceNine}{9} % experience #9
\newcommand{\adExperienceNine}{
	\cvevent{}{}{}{}
}
\newcommand{\eExperienceItemNine}{9}
\newcommand{\eExperienceNine}{}
\newcommand{\fExperienceNine}{9}
\newcommand{\eExperienceItemsNine}{\cvitems[\fExperienceNine]{}{}{}{}{}}

\newcommand{\ExperienceTen}{9} % experience #10
\newcommand{\adExperienceTen}{
	\cvevent{}{}{}{}
}
\newcommand{\eExperienceItemTen}{9}
\newcommand{\eExperienceTen}{}
\newcommand{\fExperienceTen}{9}
\newcommand{\eExperienceItemsTen}{\cvitems[\fExperienceTen]{}{}{}{}{}}

% hobbies chapter settings
\newcommand{\WithHobbies}{9} % set to 1 to include hobbies chapter
\newcommand{\Hobbies}{My Hobbies} % chapter TITLE

\newcommand{\HobbyOne}{1}
\newcommand{\aHobbyOne}{Philosophy}

\newcommand{\HobbyTwo}{1}
\newcommand{\aHobbyTwo}{Yoga}

\newcommand{\HobbyThree}{1}
\newcommand{\aHobbyThree}{Reading}

\newcommand{\HobbyFour}{9}
\newcommand{\aHobbyFour}{Cooking}

\newcommand{\HobbyFive}{1}
\newcommand{\aHobbyFive}{Woodworking}

\newcommand{\HobbySix}{9}
\newcommand{\aHobbySix}{Working \faHeart You}

\newcommand{\HobbySeven}{9}
\newcommand{\aHobbySeven}{}

\newcommand{\HobbyEight}{9}
\newcommand{\aHobbyEight}{}

\newcommand{\HobbyNine}{9}
\newcommand{\aHobbyNine}{}

\newcommand{\HobbyTen}{9}
\newcommand{\aHobbyTen}{}

\newcommand{\HobbyEleven}{9}
\newcommand{\aHobbyEleven}{}

\newcommand{\HobbyTwelve}{9}
\newcommand{\aHobbyTwelve}{}

\newcommand{\HobbyThirteen}{9}
\newcommand{\aHobbyThirteen}{}

\newcommand{\HobbyFourteen}{9}
\newcommand{\aHobbyFourteen}{}

\newcommand{\HobbyFifteen}{1}
\newcommand{\aHobbyFifteen}{Cats \href{https://github.com/Pleiadas/playacv/}{\faGithub}}

% languages chapter settings
\newcommand{\WithLanguages}{1} % set to 1 to include languages chapter
\newcommand{\Languages}{Languages} % chapter TITLE

\newcommand{\LanguageOne}{1} % set to 1 to include language
\newcommand{\aLanguageOne}{My first language} % language name
\newcommand{\bLanguageOne}{5} % language skill level from 1 to 5
\newcommand{\cLanguageOne}{native} % language level description
\newcommand{\fLanguageOne}{9} % set to 1 to highlight language

\newcommand{\LanguageTwo}{1} % language #2
\newcommand{\aLanguageTwo}{English}
\newcommand{\bLanguageTwo}{4}
\newcommand{\cLanguageTwo}{proficient}
\newcommand{\fLanguageTwo}{9}

\newcommand{\LanguageThree}{1} % language #3
\newcommand{\aLanguageThree}{Chinese}
\newcommand{\bLanguageThree}{3.5}
\newcommand{\cLanguageThree}{independent}
\newcommand{\fLanguageThree}{9}

\newcommand{\LanguageFour}{1} % language #4
\newcommand{\aLanguageFour}{Swedish}
\newcommand{\bLanguageFour}{2}
\newcommand{\cLanguageFour}{basic}
\newcommand{\fLanguageFour}{9}

\newcommand{\LanguageFive}{9} % language #5
\newcommand{\aLanguageFive}{Norwegian}
\newcommand{\bLanguageFive}{5}
\newcommand{\cLanguageFive}{native}
\newcommand{\fLanguageFive}{9}

\newcommand{\LanguageSix}{9} % language #6
\newcommand{\aLanguageSix}{Spanish}
\newcommand{\bLanguageSix}{5}
\newcommand{\cLanguageSix}{native}
\newcommand{\fLanguageSix}{9}

\newcommand{\LanguageSeven}{9} % language #7
\newcommand{\aLanguageSeven}{French}
\newcommand{\bLanguageSeven}{1}
\newcommand{\cLanguageSeven}{beginner}
\newcommand{\fLanguageSeven}{9}

\newcommand{\LanguageEight}{9} % language #8
\newcommand{\aLanguageEight}{Italian}
\newcommand{\bLanguageEight}{5}
\newcommand{\cLanguageEight}{native}
\newcommand{\fLanguageEight}{9}

\newcommand{\LanguageNine}{9} % language #9
\newcommand{\aLanguageNine}{Italian}
\newcommand{\bLanguageNine}{5}
\newcommand{\cLanguageNine}{native}
\newcommand{\fLanguageNine}{9}

\newcommand{\LanguageTen}{9} % language #10
\newcommand{\aLanguageTen}{Italian}
\newcommand{\bLanguageTen}{5}
\newcommand{\cLanguageTen}{native}
\newcommand{\fLanguageTen}{9}

% personal information chapter settings, 1 = include, 9 = NEIN
\newcommand{\WithPersonalInfo}{9} % set to 1 to include personal information chapter
\newcommand{\PersonalInfo}{Personal data} % chapter TITLE

\newcommand{\PersonalInfoOne}{1} % set to 1 to include personal detail
\newcommand{\aPersonalInfoOne}{Birth date, place} % personal information top (bold)
% See https://mirrors.ibiblio.org/CTAN/fonts/fontawesome/doc/fontawesome.pdf
\newcommand{\bPersonalInfoOne}{\faSeedling} % symbol left of personal information from fontawesome 
\newcommand{\cPersonalInfoOne}{5 Nov 2005, Hopeman, Scotland} % personal info bottom (normal)

\newcommand{\PersonalInfoTwo}{1} % personal information #2
\newcommand{\aPersonalInfoTwo}{Citizenships}
\newcommand{\bPersonalInfoTwo}{\faFlag[]}
\newcommand{\cPersonalInfoTwo}{Vietnam, UK}

\newcommand{\PersonalInfoThree}{9} % personal information #3
\newcommand{\aPersonalInfoThree}{Marital status}
\newcommand{\bPersonalInfoThree}{\faGg} % \faVenusDouble \faVenusMars \faMarsDouble
\newcommand{\cPersonalInfoThree}{Married}

\newcommand{\PersonalInfoFour}{9} % personal information #4
\newcommand{\aPersonalInfoFour}{Driving License}
\newcommand{\bPersonalInfoFour}{\faCar}
\newcommand{\cPersonalInfoFour}{B (EU)}

\newcommand{\PersonalInfoFive}{9} % personal information #5
\newcommand{\aPersonalInfoFive}{Gender, pronouns}
\newcommand{\bPersonalInfoFive}{\faGenderless} % \faVenus \faTransgender \faIntersex \faTransgenderAlt \faMars
\newcommand{\cPersonalInfoFive}{you} % or hem. Whatever suits you

\newcommand{\PersonalInfoSix}{9} % personal information #6
\newcommand{\aPersonalInfoSix}{Disabilities}
\newcommand{\bPersonalInfoSix}{\faUniversalAccess}
\newcommand{\cPersonalInfoSix}{}

\newcommand{\PersonalInfoSeven}{9} % personal information #7
\newcommand{\aPersonalInfoSeven}{Something else}
\newcommand{\bPersonalInfoSeven}{\faFeather*}
\newcommand{\cPersonalInfoSeven}{The personal detail}

\newcommand{\PersonalInfoEight}{9} % personal information #8
\newcommand{\aPersonalInfoEight}{Residency permit in}
\newcommand{\bPersonalInfoEight}{\faFiles[]} % see https://texdoc.org/serve/fontawesome5/0 for symbols
\newcommand{\cPersonalInfoEight}{Something else}

\newcommand{\PersonalInfoNine}{9} % personal information #9
\newcommand{\aPersonalInfoNine}{Something}
\newcommand{\bPersonalInfoNine}{\faGg}
\newcommand{\cPersonalInfoNine}{Something something}

\newcommand{\PersonalInfoTen}{9} % personal information #10
\newcommand{\aPersonalInfoTen}{Something you need to share}
\newcommand{\bPersonalInfoTen}{\faGg}
\newcommand{\cPersonalInfoTen}{Some juicy data}

% projects chapter settings
\newcommand{\WithProjects}{9} % set to 1 to include projects chapter
\newcommand{\Projects}{Projects} % chapter TITLE

\newcommand{\ProjectOne}{1} % set to 1 to include project entry
\newcommand{\aProjectOne}{Last Project} % project title
\newcommand{\bProjectOne}{Role I had} % project bold subtitle
\newcommand{\cProjectOne}{Location} % place
\newcommand{\dProjectOne}{2023 -- 2024} % date
\newcommand{\eProjectOne}{} % text

\newcommand{\ProjectTwo}{1} % project #2
\newcommand{\aProjectTwo}{Project you might like \hfill \textbf{Institution}}
\newcommand{\bProjectTwo}{My position then}
\newcommand{\cProjectTwo}{Location}
\newcommand{\dProjectTwo}{6 months}
\newcommand{\eProjectTwo}{Description}

\newcommand{\ProjectThree}{9} % project #3
\newcommand{\aProjectThree}{Project 3}
\newcommand{\bProjectThree}{Role, info}
\newcommand{\cProjectThree}{Location}
\newcommand{\dProjectThree}{Time}
\newcommand{\eProjectThree}{Description}

\newcommand{\ProjectFour}{9} % project #4
\newcommand{\aProjectFour}{Project 4}
\newcommand{\bProjectFour}{}
\newcommand{\cProjectFour}{Location}
\newcommand{\dProjectFour}{Time}
\newcommand{\eProjectFour}{Description}

\newcommand{\ProjectFive}{9} % project #5
\newcommand{\aProjectFive}{Project 5}
\newcommand{\bProjectFive}{}
\newcommand{\cProjectFive}{Location}
\newcommand{\dProjectFive}{Time}
\newcommand{\eProjectFive}{Description}

% publications chapter settings
\newcommand{\WithPublications}{9} % set to 1 to include publications chapter
\newcommand{\Publications}{Publications} % chapter TITLE

\newcommand{\PublicationOne}{1} % set to 1 to include publication
\newcommand{\aPublicationOne}{Important publication on the subject} % optional plain left aligned title
\newcommand{\bPublicationOne}{Publisher} % optional smaller bold right aligned title text
\newcommand{\cPublicationOne}{\myAddress} % place
\newcommand{\dPublicationOne}{\today} % date
\newcommand{\ePublicationItemOne}{1} % set to 1 to use items
\newcommand{\ePublicationOne}{The highlights of the findings} % text
\newcommand{\fPublicationOne}{2} % set to 1..5 to highlight publication/item
\newcommand{\ePublicationItemsOne}{
	\cvitems[\fPublicationOne]{Detail 1}{Detail 2}{Detail 3}{}{}
}

\newcommand{\PublicationTwo}{1} % publication #2
\newcommand{\aPublicationTwo}{Publication 2}
\newcommand{\bPublicationTwo}{Publisher/institution}
\newcommand{\cPublicationTwo}{City, State}
\newcommand{\dPublicationTwo}{20YY}
\newcommand{\ePublicationItemTwo}{9} % set to 1 to use items
\newcommand{\ePublicationTwo}{Description}
\newcommand{\fPublicationTwo}{9}
\newcommand{\ePublicationItemsTwo}{
	\cvitems[\fPublicationTwo]{item 1}{item 2}{item 3}{item 4}{item 5}
}

\newcommand{\PublicationThree}{9} % publication #3
\newcommand{\aPublicationThree}{Publication 3}
\newcommand{\bPublicationThree}{Publisher/institution}
\newcommand{\cPublicationThree}{City, State}
\newcommand{\dPublicationThree}{20YY}
\newcommand{\ePublicationItemThree}{9}
\newcommand{\ePublicationThree}{}
\newcommand{\fPublicationThree}{9}
\newcommand{\ePublicationItemsThree}{
	\cvitems[\fPublicationThree]{item 1}{item 2}{}{}{}
}

\newcommand{\PublicationFour}{9} % publication #4
\newcommand{\aPublicationFour}{Publication 4}
\newcommand{\bPublicationFour}{Publisher/institution}
\newcommand{\cPublicationFour}{City, State}
\newcommand{\dPublicationFour}{20YY}
\newcommand{\ePublicationItemFour}{1}
\newcommand{\ePublicationFour}{A short bullet list}
\newcommand{\fPublicationFour}{9}
\newcommand{\ePublicationItemsFour}{
	\cvitems[\fPublicationFour]{item 1}{item 2}{}{}{}
}

\newcommand{\PublicationFive}{9} % publication #5
\newcommand{\aPublicationFive}{Publication 5}
\newcommand{\bPublicationFive}{Publisher/institution}
\newcommand{\cPublicationFive}{City, State}
\newcommand{\dPublicationFive}{20YY}
\newcommand{\ePublicationItemFive}{9}
\newcommand{\ePublicationFive}{A lone abstract would also work.}
\newcommand{\fPublicationFive}{9}
\newcommand{\ePublicationItemsFive}{
	\cvitems[\fPublicationFive]{}{}{}{}{}
}

% references chapter settings
\newcommand{\WithReferences}{9} % set to 1 to include references chapter
\newcommand{\References}{References} % chapter TITLE
\newcommand{\ReferenceText}{Further references and supporting \\documentation available upon request} % text at bottom of references chapter

\newcommand{\ReferenceOne}{1} % set to 1 to include reference entry
\newcommand{\fReferenceOne}{9} % set to 1 to highlight reference, will appear with "Suggested:" tag
\newcommand{\aReferenceOne}{ 
	\cvref[\fReferenceOne]
	{Judith Referee} % name of reference contact
	{Company} % company
	{mail@refer.ee} % full mail address of reference
	{mail@referee.com} % text of mail address
	{} % second line e.g. for long email addresses
	{+42 123 45678} % phone number of reference
	{Department, role} % role or other data of reference
	{City, State} % location of reference 
}

\newcommand{\ReferenceTwo}{1} % reference #2
\newcommand{\fReferenceTwo}{9}
\newcommand{\aReferenceTwo}{ 
	\cvref[\fReferenceTwo]
	{Mark Reference}
	{Microsoft Corporation}
	{mark.reference@company.www}
	{mark.reference@company.www}
	{US ET time zone} % free line without symbol
	{+1 234 567890} % phone number of reference
	{ChatGPT commerce, former lead} % role or other data of reference
	{Redmond, Washington (US)} % location of reference 
}

\newcommand{\ReferenceThree}{9} % reference #3
\newcommand{\fReferenceThree}{9}
\newcommand{\aReferenceThree}{ 
	\cvref[\fReferenceThree]
	{Monica Reference}
	{}
	{monica@reference.cv}
	{monica@reference.cv}
	{} % free line
	{+1 111 222 33}
	{HR department, former manager} % role or other data of reference
	{} % location of reference 
}

\newcommand{\ReferenceFour}{9} % reference #4
\newcommand{\fReferenceFour}{9}
\newcommand{\aReferenceFour}{ 
	\cvref[\fReferenceFour]
	{Mark Reference}
	{Microsoft Corporation}
	{mark.reference@company.www}
	{US ET time zone}
	{+1 234 567890} % phone number of reference
	{ChatGPT commerce, former head} % role or other data of reference
	{Redmond, Washington (US)} % location of reference 
}

\newcommand{\ReferenceFive}{9} % reference #5
\newcommand{\fReferenceFive}{9}
\newcommand{\aReferenceFive}{ 
	\cvref[\fReferenceFive]
	{Mark Reference}
	{Microsoft Corporation}
	{mark.reference@company.www}
	{US ET time zone}
	{+1 234 567890} % phone number of reference
	{ChatGPT commerce, former head} % role or other data of reference
	{Redmond, Washington (US)} % location of reference 
}

% hard skills chapter settings
\newcommand{\WithSkills}{1} % set to 1 to include hard skills chapter
\newcommand{\HardSkills}{Skills} % chapter TITLE

\newcommand{\SkillOne}{1} % set to 1 to include skill with scale
\newcommand{\aSkillOne}{A hard skill} % skill name
\newcommand{\bSkillOne}{4.5} % skill level from 1 to 5
\newcommand{\fSkillOne}{9} % set to 1 to highlight skill

\newcommand{\SkillTwo}{1} % skill #2
\newcommand{\aSkillTwo}{For example}
\newcommand{\bSkillTwo}{4}
\newcommand{\fSkillTwo}{9}

\newcommand{\SkillThree}{1} % skill #3
\newcommand{\aSkillThree}{Design thinking}
\newcommand{\bSkillThree}{3.5}
\newcommand{\fSkillThree}{9}

\newcommand{\SkillFour}{1} % skill #4
\newcommand{\aSkillFour}{PCB design}
\newcommand{\bSkillFour}{4}
\newcommand{\fSkillFour}{9}

\newcommand{\SkillFive}{1} % skill #5
\newcommand{\aSkillFive}{Microsoft Office}
\newcommand{\bSkillFive}{4.5}
\newcommand{\fSkillFive}{9}

\newcommand{\SkillSix}{9} % skill #6
\newcommand{\aSkillSix}{Sikill six}
\newcommand{\bSkillSix}{5}
\newcommand{\fSkillSix}{9}

\newcommand{\SkillSeven}{9} % skill #7
\newcommand{\aSkillSeven}{Skill seven}
\newcommand{\bSkillSeven}{3}
\newcommand{\fSkillSeven}{9}

\newcommand{\SkillEight}{9} % skill #8
\newcommand{\aSkillEight}{Skill eight}
\newcommand{\bSkillEight}{4.5}
\newcommand{\fSkillEight}{9}

\newcommand{\SkillNine}{9} % skill #9
\newcommand{\aSkillNine}{Skill nine}
\newcommand{\bSkillNine}{5}
\newcommand{\fSkillNine}{9}

\newcommand{\SkillTen}{9} % skill #10
\newcommand{\aSkillTen}{\textcolor{Grey}{\textmd{\textsl{\Large \LaTeX}}}}
\newcommand{\bSkillTen}{5}
\newcommand{\fSkillTen}{9}

% soft skills chapter settings
\newcommand{\WithSoftSkills}{9} % set to 1 to include soft skills chapter
\newcommand{\SoftSkills}{Soft skills} % chapter TITLE
\newcommand{\SoftSkillsRated}{1} % set to 1 to use rated soft skills

\newcommand{\SoftSkillOne}{1} % set to 1 to include skill as tag without scale
\newcommand{\aSoftSkillOne}{Softskil1} % skill name, use urls with \href{http://link.com}{tag} for clickable tags
%\newcommand{\bSoftSkillOne}{2.5} % add skill level from 1 to 5 if you change to \cvskill in _SoftSkills.tex
\newcommand{\fSoftSkillOne}{9} % set to 1 to highlight soft skill

\newcommand{\SoftSkillTwo}{1} % soft skill #2
\newcommand{\aSoftSkillTwo}{Good at 2}
%\newcommand{\bSoftSkillTwo}{5}
\newcommand{\fSoftSkillTwo}{9}

\newcommand{\SoftSkillThree}{1} % soft skill #3
\newcommand{\aSoftSkillThree}{Take me for 3}
%\newcommand{\bSoftSkillThree}{5}
\newcommand{\fSoftSkillThree}{9}

\newcommand{\SoftSkillFour}{1} % soft skill #4
\newcommand{\aSoftSkillFour}{Problem solving}
%\newcommand{\bSoftSkillFour}{5}
\newcommand{\fSoftSkillFour}{9}

\newcommand{\SoftSkillFive}{1} % soft skill #5
\newcommand{\aSoftSkillFive}{Result orientated}
%\newcommand{\bSoftSkillFive}{5}
\newcommand{\fSoftSkillFive}{9}

\newcommand{\SoftSkillSix}{9} % soft skill #6
\newcommand{\aSoftSkillSix}{Respectful communication}
\newcommand{\fSoftSkillSix}{9}

\newcommand{\SoftSkillSeven}{9} % soft skill #7
\newcommand{\aSoftSkillSeven}{Inclusive}
\newcommand{\fSoftSkillSeven}{9}

\newcommand{\SoftSkillEight}{9} % soft skill #8
\newcommand{\aSoftSkillEight}{Teamwork}
\newcommand{\fSoftSkillEight}{9}

\newcommand{\SoftSkillNine}{9} % soft skill #9
\newcommand{\aSoftSkillNine}{Fact-based decision making}
\newcommand{\fSoftSkillNine}{9}

\newcommand{\SoftSkillTen}{9} % soft skill #10
\newcommand{\aSoftSkillTen}{Creativity}
\newcommand{\fSoftSkillTen}{9}

\newcommand{\SoftSkillEleven}{9} % soft skill #11
\newcommand{\aSoftSkillEleven}{aSoftSkill}
\newcommand{\fSoftSkillEleven}{9}

\newcommand{\SoftSkillTwelve}{9} % soft skill #12
\newcommand{\aSoftSkillTwelve}{SoftSkill}
\newcommand{\fSoftSkillTwelve}{9}

\newcommand{\SoftSkillThirteen}{9} % soft skill #13
\newcommand{\aSoftSkillThirteen}{oftSkill}
\newcommand{\fSoftSkillThirteen}{9}

\newcommand{\SoftSkillFourteen}{9} % soft skill #14
\newcommand{\aSoftSkillFourteen}{ftSkill}
\newcommand{\fSoftSkillFourteen}{9}

\newcommand{\SoftSkillFifteen}{9} % soft skill #15
\newcommand{\aSoftSkillFifteen}{tSkill}
\newcommand{\fSoftSkillFifteen}{9}

% volunteering chapter settings
\newcommand{\WithSocial}{9} % set to 1 to include volunteering chapter
\newcommand{\Social}{Volunteering} % chapter TITLE

\newcommand{\SocialOne}{1} % set to 1 to include volunteering entry
\newcommand{\adSocialOne}{
	\cvevent{My honorary post \hfill \small{\textbf{Volunteering}}}
	{Subtitle if needed}
	{Location}
	{Mon 20YY -- now}
}
\newcommand{\eSocialItemOne}{1} % set to 1 to show items
\newcommand{\eSocialOne}{Description if needed} % text (above items)
\newcommand{\fSocialOne}{9} % set to 1..5 to highlight volunteering entry item
\newcommand{\eSocialItemsOne}{\cvitems[\fSocialOne] % items{1}..{5}
	{items available}
	{}{}{}{}
}

\newcommand{\SocialTwo}{1} % volunteering #2
\newcommand{\adSocialTwo}{\cvevent{Volunteering 2}
	{}
	{Location}
	{Time}
}
\newcommand{\eSocialItemTwo}{9}
\newcommand{\eSocialTwo}{
	Activity text
}
\newcommand{\fSocialTwo}{9}
\newcommand{\eSocialItemsTwo}{
	\cvitems[\fSocialTwo]{}{}{}{}{}
}

\newcommand{\SocialThree}{9} % volunteering #3
\newcommand{\adSocialThree}{\cvevent{}
	{Volunteering 3 small title}
	{Location}
	{Time}}
\newcommand{\eSocialItemThree}{1}
\newcommand{\eSocialThree}{}
\newcommand{\fSocialThree}{1}
\newcommand{\eSocialItemsThree}{
	\cvitems[\fSocialThree]
	{activity item}
	{}{}{}{}
}

\newcommand{\SocialFour}{9} % volunteering #4
\newcommand{\adSocialFour}{\cvevent{Volunteering 4}
	{Volunteering 4 subtitle}
	{Location}
	{Oct 2013 -- Sep 2015}}
\newcommand{\eSocialItemFour}{1}
\newcommand{\eSocialFour}{}
\newcommand{\fSocialFour}{9}
\newcommand{\eSocialItemsFour}{
	\cvitems[\fSocialFour]
	{}{}{}{}{}}

\newcommand{\SocialFive}{9} % volunteering #5
\newcommand{\adSocialFive}{\cvevent{Volunteering 5}
	{Volunteering 5 subtitle}
	{everywhere}
	{always}}
\newcommand{\eSocialItemFive}{1}
\newcommand{\eSocialFive}{anything can help}
\newcommand{\fSocialFive}{9}
\newcommand{\eSocialItemsFive}{\cvitems[\fSocialFive]{Good deed 1}{}{}{}{}}

% summary chapter settings
\newcommand{\WithSummary}{9} % set to 1 to include summary chapter
\newcommand{\Summary}{Summary} % chapter TITLE

\newcommand{\aSummary}{ % summary text using \cvbox
	This is a short summary on why you are perfect for this position.\\
	\smallskip
	Useful for applications where no cover letter is required
}
\newcommand{\bSummary}{black} % summary text color
\newcommand{\cSummary}{SymbolColor!10} % summary background fill color
\newcommand{\dSummary}{0.9} % width in 0..1 \linewidth
\newcommand{\eSummary}{center} % text alignment, options left, center, right, justify etc.

% user chapter settings
\newcommand{\WithUser}{9} % set to 1 to include user defined chapter
\newcommand{\User}{My chapter} % chapter TITLE

\newcommand{\UserOne}{1} % set to 1 to include user chapter cv element #1
\newcommand{\aUserOne}{
	% Adapted from @Jake's answer from http://tex.stackexchange.com/a/82729/226
	% \wheelchart{outer radius}{inner radius}{
		% comma-separated list of value/text width/color/detail}
	\wheelchart{1.5cm}{0.5cm}{%
		6/8em/accent!30/{Sleep, beautiful sleep},
		3/8em/accent!40/Hopeful inventor by night,
		8/8em/accent!60/Daytime job,
		2/10em/accent/Sports and relaxation,
		5/6em/accent!20/Spending time with family
	}
} % a custom user entry

% Recipient information
\newcommand{\RecipientName}{Sabrina Hiring} % name (e.g. "Sabrina Hiring")
\newcommand{\Company}{Microsoft Corporation} % company (e.g. "Microsoft Corporation")
\newcommand{\Subject}{Application for dream job 12345} % company (e.g. "Microsoft Corporation")
\newcommand{\Greeting}{Dear} % greeting to use (e.g. "Dear")
\newcommand{\Closer}{Kind Regards} % closer to use (e.g. "Kind Regards")
\newcommand{\CompanyStreet}{1 Microsoft Way} % company's street address (e.g. "1 Microsoft Way")
\newcommand{\CompanyCity}{Redmond} % ompany's city (e.g. "Redmond")
\newcommand{\CompanyState}{WA} % company's state prefix (e.g. "WA")
\newcommand{\CompanyZip}{98052} % company's zip code (e.g. "98052")